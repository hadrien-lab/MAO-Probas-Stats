\documentclass[a4paper, 10pt, oneside]{article}
\renewcommand{\contentsname}{Contenu}
\usepackage[utf8]{inputenc}
\pagenumbering{arabic}
\renewcommand*\sfdefault{ugq}
\usepackage[T1]{fontenc}
\usepackage[french]{babel}
\usepackage{datetime}  
\usepackage{lmodern}
\usepackage[
  top=0.5in,
  bottom=0.7in,
  left=0.7in,
  right=0.7in,
  headheight=17pt, % as per the warning by fancyhdr
  includehead,includefoot,
  heightrounded, % to avoid spurious underfull messages
]{geometry} 
\renewcommand{\baselinestretch}{1.5}
\usepackage[some]{background}
\usepackage{lipsum}
\usepackage{amsmath}
\usepackage{amssymb}
\usepackage{amsfonts}
\usepackage{multicol}
\usepackage{mathrsfs}
\usepackage{color}
\usepackage{enumitem}
\usepackage{xcolor}
\usepackage{graphicx}
\usepackage{titlesec}
\usepackage{amsmath,esint}
\usepackage{tikz}
\usepackage{pgfplots}
\pgfplotsset{compat=1.16} 
\usepackage[framemethod=tikz]{mdframed}
\usepackage{xpatch}
\usepackage{marvosym}
\usepackage{amsthm}
\usepackage[justification=centering]{caption}
\usepackage[all]{xy}
\usepackage{empheq}
\usepackage{mathrsfs}
\usetikzlibrary{automata,topaths,arrows.meta,calc,patterns,angles,quotes,positioning,graphs,graphs.standard,positioning,chains,fit,shapes,calc}
%% The following commands all in one way or another set us up to be able to draw graphs.
%%
%% The calc package is used for calculating angles to evenly space vertices in circular arrangements.
\usepackage{calc}
%%
%% The tikz package is used for doing the actual drawing.
\usepackage{tikz}
%%
%% In order to be able to put arrowheads in the middle of directed edges, we need an extra library.
\usetikzlibrary{decorations.markings}
%%
%% The next line says how the "vertex" style of nodes should look: drawn as small circles.
\tikzstyle{vertex}=[circle, draw, inner sep=0pt, minimum size=6pt]
%%
%% Next, we make a \vertex command as a shorthand in place of \node[vertex} to get that style.
\newcommand{\vertex}{\node[vertex]}
%%
%% Finally, we declare a "counter", which is what LaTeX calls an integer variable, for use in
%% the calculations of angles for evenly spacing vertices in circular arrangements.
\newcounter{Angle}

\usepackage{hyperref}
\usepackage{comment}
\usepackage{fancyhdr}
\usepackage{courier}
\usepackage[most]{tcolorbox}
\usepackage[colorinlistoftodos]{todonotes}
\usepackage{lcg}
\newcommand{\random}{\rand\arabic{rand}}

\usepackage{attachfile}
\usepackage{navigator}
\usepackage{float}
\usepackage{eso-pic}

\usepackage{pdfpages}

\usepackage{listings}
\lstset{
    language=Python,
    basicstyle=\ttfamily\small,
    aboveskip={1.0\baselineskip},
    belowskip={1.0\baselineskip},
    columns=fixed,
    extendedchars=true,
    breaklines=true,
    tabsize=4,
    prebreak=\raisebox{0ex}[0ex][0ex]{\ensuremath{\hookleftarrow}},
    frame=none,
    showtabs=false,
    showspaces=false,
    showstringspaces=false,
    keywordstyle=\color[rgb]{0.627,0.126,0.941},
    commentstyle=\color[rgb]{0.133,0.545,0.133},
    stringstyle=\color[rgb]{01,0,0},
    numbers=left,
    numberstyle=\small,
    stepnumber=1,
    numbersep=10pt,
    captionpos=t,
    escapeinside={\%*}{*)}
}

\pagestyle{fancy}
\renewcommand{\headrulewidth}{0pt}

\fancyhead[L,R]{}
\fancyhead[L,R]{}
\fancyfoot[C]{\thepage}

\newtcolorbox{activitybox}[1][]{%
    breakable,
    enhanced,
    colback=white,
    colframe=black,
    coltitle=white,
    #1
}

\newcommand*{\fakebreak}{\par\vspace{\textheight minus \textheight}\pagebreak}
\newcommand\bord{\boldsymbol{B}}
\newcommand\lapla{\boldsymbol{\mathcal{L}}}

\usepackage{etoolbox}
\patchcmd{\thebibliography}{\section*{\refname}}{}{}{}

\definecolor{blueee}{rgb}{0.2, 0.2, 0.6}

\hypersetup{
colorlinks = true,
allcolors = blueee,
linkbordercolor = {white},
citecolor = black,}


\renewcommand{\thesubsection}{\arabic{subsection}}

\begin{document}
\definecolor{myblue}{RGB}{80,80,160}
\definecolor{mygreen}{RGB}{80,160,80}


\newtheorem{theo}{Théorème}
\newtheorem{prop}[theo]{Proposition}
\newtheorem{coro}[theo]{Corollaire}
\newtheorem{lemme}[theo]{Lemme
}
\theoremstyle{definition}
\newtheorem{defi}{Définition}
\newtheorem*{rema}{Remarque}
\newtheorem*{exem}{Exemple}
\newtheorem*{preuve}{Preuve}
\newtheorem*{rappel}{Rappel}
\newtheorem*{appli}{Application}

\section*{TD1 : Probabilités, rappels sur les notions de base}

\subsection{Une tribu est-elle stable par intersections dénombrables ?}

Par stabilité de la tribu par union dénombrable et par passage au complémentaire :
$$
\forall(A_n)\in\mathcal{F}^\mathbf{N},\qquad\bigcap_{n\in\mathbf{N}}A_n=\overline{\bigcup_{n\in\mathbf{N}}\overline{A_n}}\in\mathcal{F} 
$$

\subsection{On répartit uniformément \texorpdfstring{$n$}{TEXT} boules dans \texorpdfstring{$d$}{TEXT} tiroirs. Quelle est la probabilité que tous les tiroirs soient occupés ?}

$\Omega=\{1\ldots d\}^n\qquad \mathcal{F}=\mathcal{P}(\Omega)\qquad\mathbb{P}=\text{équiprobabilité}$\\


\textbf{Première solution : combinatoire}

$A=\{\text{fonctions }\{1\ldots n\}\xrightarrow[]{}\{1\ldots d\}\text{ surjectives}\}$\\

$\mathbb{P}(A)=\frac{\text{card}(A)}{\text{card}(\Omega)}=\frac{\sum_{k=0}^d(-1)^{d-k}{d\choose k} k^n}{d^n}$\\

\textbf{Deuxième solution : probas}

$A_j=$"le tiroir $j$ est vide".\\

$A=\bigcap_{j=1}^d\complement_{A_j}=\complement_{\bigcup_{j=1}^dA_j}$\\

\hbox{
\noindent\textbf{Formule du crible de Poincaré :} $\mathbb{P}\left(\bigcup_{j=1}^dA_j\right)=\sum_{r=1}^d(-1)^{r-1}\sum_{1\le j_1<\ldots<j_n\le d}\mathbb{P}(A_{j_1}\cap\ldots\cap A_{j_r})$\\
Moyen mnémotechnique : \\

\begin{center}
\begin{tikzpicture}
    \draw (0,0) circle (1);
    \draw (1.1,0.75) circle (1);
    \draw (1.1,-0.75) circle (1);
    \draw (2.2,0) circle (1);
\end{tikzpicture}
\end{center}
$\mathbb{P}(A_1)+\mathbb{P}(A_2)+\mathbb{P}(A_3)+\mathbb{P}(A_4)$, on compte les intersections en trop.
}

\begin{rema}
On peut imaginer une preuve utilisant la loi multinomial.
\end{rema}

\subsection{Exercice 3 : Soit \texorpdfstring{$n$}{TEXT} boules dans \texorpdfstring{$(A_n)_{n\in\mathbf{N}}$}{TEXT} une suite d'événements, montrer que \texorpdfstring{$n$}{TEXT} boules dans \texorpdfstring{$\mathbb{P}(\limsup A_n)=1$}{TEXT} si et seulement si \texorpdfstring{$\sum_{n=1}^{+\infty}\mathbb{P}(A\cap A_n)=+\infty$}{TEXT} pour tout événement \texorpdfstring{$A$}{TEXT} de probabilité strictement positive.}

Montrons $\Rightarrow$ par contraposée. On suppose qu'il existe un événement $A$ de probabilité strictement positive telle que $\sum_{n=1}^{+\infty}\mathbb{P}(A\cap A_n)<+\infty$. Par Borel-Cantelli 1 : $\mathbb{P}(\limsup A\cap A_n)=0$, or $\mathbb{P}(\limsup A\cap A_n)=\mathbb{P}(A\cap\limsup A_n)$, donc cette dernière quantité est nulle. Par l'absurde, si $\mathbb{P}(\limsup A_n)=1$ alors $\mathbb{P}(A\cap \limsup A_n)=\mathbb{P}(A)>0$, ce qui est contradictoire. En conclusion, $\mathbb{P}(\limsup A_n)<1$.




\end{document}
